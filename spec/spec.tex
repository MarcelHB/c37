\documentclass[11pt,a4paper,notitlepage]{report}

\usepackage[margin=2cm]{geometry}
\usepackage{lmodern}
\usepackage{color}
\usepackage[latin1]{inputenc}

\begin{document}
	\begin{center}
		TU Hamburg-Harburg -- Prozedurale Programmierung \\
		Gruppe \#37: Florian J******, Marc L******, Marcel H******, WS 2010/2011
	\end{center}
	\rule{1.0\linewidth}{.1pt}
	\newline
	\begin{center}
		Das C-Projekt:
		\Huge
		\parbox{1.0\linewidth}{
			\center{Rogue-like-like?}
		}
		\normalsize
	\end{center}
	\begin{center}
		\textcolor{red}{Entwurf!}
	\end{center}
	
	\section*{Spezifikation}
		\begin{itemize}
			\item 2D-Adventure
			\item ASCII-Zeichen / Farben zur Darstellung (vgl. \textit{Rogue-like})
			\item Steuerung einer Spielfigur durch Dungeons und H�hlen per Tastatur
			\item Verschiedene Karten spielbar
			\item Weitestgehend Platformunabh�ngig (SDL, SDL\_ttf, ...?)
			\item ...
		\end{itemize}
		
	\section*{Design}
		\begin{itemize}
			\item Map-Loader (JSON, TXT, XML, BIN, ...?)
			\item Ausgabe-Wrapper (SDL, Konsole, OpenGL, ...?)
			\item logische Layer (Map -$>$ Logik -$>$ Ausgabepuffer -$>$ Ausgabe)
			\item Tastatur-Listener
			\item ...
		\end{itemize}
	
	\section*{Aufgaben und Zeitplan}

	\begin{tabular}{ p{3.5cm} | p{4.0cm} | p{5.0cm} | p{2.5cm} }
		Wer? & Arbeitspaket & T�tigkeiten & Datum \\
		\hline
		Marcel & Planung & Demos (SDL\_ttf, JSON) & 18. + 19.12.10
	\end{tabular}

\end{document}
